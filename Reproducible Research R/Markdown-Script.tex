% Options for packages loaded elsewhere
\PassOptionsToPackage{unicode}{hyperref}
\PassOptionsToPackage{hyphens}{url}
%
\documentclass[
]{article}
\usepackage{amsmath,amssymb}
\usepackage{iftex}
\ifPDFTeX
  \usepackage[T1]{fontenc}
  \usepackage[utf8]{inputenc}
  \usepackage{textcomp} % provide euro and other symbols
\else % if luatex or xetex
  \usepackage{unicode-math} % this also loads fontspec
  \defaultfontfeatures{Scale=MatchLowercase}
  \defaultfontfeatures[\rmfamily]{Ligatures=TeX,Scale=1}
\fi
\usepackage{lmodern}
\ifPDFTeX\else
  % xetex/luatex font selection
\fi
% Use upquote if available, for straight quotes in verbatim environments
\IfFileExists{upquote.sty}{\usepackage{upquote}}{}
\IfFileExists{microtype.sty}{% use microtype if available
  \usepackage[]{microtype}
  \UseMicrotypeSet[protrusion]{basicmath} % disable protrusion for tt fonts
}{}
\makeatletter
\@ifundefined{KOMAClassName}{% if non-KOMA class
  \IfFileExists{parskip.sty}{%
    \usepackage{parskip}
  }{% else
    \setlength{\parindent}{0pt}
    \setlength{\parskip}{6pt plus 2pt minus 1pt}}
}{% if KOMA class
  \KOMAoptions{parskip=half}}
\makeatother
\usepackage{xcolor}
\usepackage[margin=1in]{geometry}
\usepackage{color}
\usepackage{fancyvrb}
\newcommand{\VerbBar}{|}
\newcommand{\VERB}{\Verb[commandchars=\\\{\}]}
\DefineVerbatimEnvironment{Highlighting}{Verbatim}{commandchars=\\\{\}}
% Add ',fontsize=\small' for more characters per line
\usepackage{framed}
\definecolor{shadecolor}{RGB}{248,248,248}
\newenvironment{Shaded}{\begin{snugshade}}{\end{snugshade}}
\newcommand{\AlertTok}[1]{\textcolor[rgb]{0.94,0.16,0.16}{#1}}
\newcommand{\AnnotationTok}[1]{\textcolor[rgb]{0.56,0.35,0.01}{\textbf{\textit{#1}}}}
\newcommand{\AttributeTok}[1]{\textcolor[rgb]{0.13,0.29,0.53}{#1}}
\newcommand{\BaseNTok}[1]{\textcolor[rgb]{0.00,0.00,0.81}{#1}}
\newcommand{\BuiltInTok}[1]{#1}
\newcommand{\CharTok}[1]{\textcolor[rgb]{0.31,0.60,0.02}{#1}}
\newcommand{\CommentTok}[1]{\textcolor[rgb]{0.56,0.35,0.01}{\textit{#1}}}
\newcommand{\CommentVarTok}[1]{\textcolor[rgb]{0.56,0.35,0.01}{\textbf{\textit{#1}}}}
\newcommand{\ConstantTok}[1]{\textcolor[rgb]{0.56,0.35,0.01}{#1}}
\newcommand{\ControlFlowTok}[1]{\textcolor[rgb]{0.13,0.29,0.53}{\textbf{#1}}}
\newcommand{\DataTypeTok}[1]{\textcolor[rgb]{0.13,0.29,0.53}{#1}}
\newcommand{\DecValTok}[1]{\textcolor[rgb]{0.00,0.00,0.81}{#1}}
\newcommand{\DocumentationTok}[1]{\textcolor[rgb]{0.56,0.35,0.01}{\textbf{\textit{#1}}}}
\newcommand{\ErrorTok}[1]{\textcolor[rgb]{0.64,0.00,0.00}{\textbf{#1}}}
\newcommand{\ExtensionTok}[1]{#1}
\newcommand{\FloatTok}[1]{\textcolor[rgb]{0.00,0.00,0.81}{#1}}
\newcommand{\FunctionTok}[1]{\textcolor[rgb]{0.13,0.29,0.53}{\textbf{#1}}}
\newcommand{\ImportTok}[1]{#1}
\newcommand{\InformationTok}[1]{\textcolor[rgb]{0.56,0.35,0.01}{\textbf{\textit{#1}}}}
\newcommand{\KeywordTok}[1]{\textcolor[rgb]{0.13,0.29,0.53}{\textbf{#1}}}
\newcommand{\NormalTok}[1]{#1}
\newcommand{\OperatorTok}[1]{\textcolor[rgb]{0.81,0.36,0.00}{\textbf{#1}}}
\newcommand{\OtherTok}[1]{\textcolor[rgb]{0.56,0.35,0.01}{#1}}
\newcommand{\PreprocessorTok}[1]{\textcolor[rgb]{0.56,0.35,0.01}{\textit{#1}}}
\newcommand{\RegionMarkerTok}[1]{#1}
\newcommand{\SpecialCharTok}[1]{\textcolor[rgb]{0.81,0.36,0.00}{\textbf{#1}}}
\newcommand{\SpecialStringTok}[1]{\textcolor[rgb]{0.31,0.60,0.02}{#1}}
\newcommand{\StringTok}[1]{\textcolor[rgb]{0.31,0.60,0.02}{#1}}
\newcommand{\VariableTok}[1]{\textcolor[rgb]{0.00,0.00,0.00}{#1}}
\newcommand{\VerbatimStringTok}[1]{\textcolor[rgb]{0.31,0.60,0.02}{#1}}
\newcommand{\WarningTok}[1]{\textcolor[rgb]{0.56,0.35,0.01}{\textbf{\textit{#1}}}}
\usepackage{longtable,booktabs,array}
\usepackage{calc} % for calculating minipage widths
% Correct order of tables after \paragraph or \subparagraph
\usepackage{etoolbox}
\makeatletter
\patchcmd\longtable{\par}{\if@noskipsec\mbox{}\fi\par}{}{}
\makeatother
% Allow footnotes in longtable head/foot
\IfFileExists{footnotehyper.sty}{\usepackage{footnotehyper}}{\usepackage{footnote}}
\makesavenoteenv{longtable}
\usepackage{graphicx}
\makeatletter
\newsavebox\pandoc@box
\newcommand*\pandocbounded[1]{% scales image to fit in text height/width
  \sbox\pandoc@box{#1}%
  \Gscale@div\@tempa{\textheight}{\dimexpr\ht\pandoc@box+\dp\pandoc@box\relax}%
  \Gscale@div\@tempb{\linewidth}{\wd\pandoc@box}%
  \ifdim\@tempb\p@<\@tempa\p@\let\@tempa\@tempb\fi% select the smaller of both
  \ifdim\@tempa\p@<\p@\scalebox{\@tempa}{\usebox\pandoc@box}%
  \else\usebox{\pandoc@box}%
  \fi%
}
% Set default figure placement to htbp
\def\fps@figure{htbp}
\makeatother
\setlength{\emergencystretch}{3em} % prevent overfull lines
\providecommand{\tightlist}{%
  \setlength{\itemsep}{0pt}\setlength{\parskip}{0pt}}
\setcounter{secnumdepth}{-\maxdimen} % remove section numbering
\usepackage{bookmark}
\IfFileExists{xurl.sty}{\usepackage{xurl}}{} % add URL line breaks if available
\urlstyle{same}
\hypersetup{
  pdftitle={Análisis de Datos de Monitoreo de Actividad Personal},
  pdfauthor={Proyecto de Investigación Reproducible},
  hidelinks,
  pdfcreator={LaTeX via pandoc}}

\title{Análisis de Datos de Monitoreo de Actividad Personal}
\author{Proyecto de Investigación Reproducible}
\date{2025-10-02}

\begin{document}
\maketitle

\subsection{Introducción}\label{introducciuxf3n}

Es posible recolectar una gran cantidad de datos sobre el movimiento
personal usando dispositivos de monitoreo de actividad como
\href{http://www.fitbit.com/}{Fitbit},
\href{http://www.nike.com/us/en_us/c/nikeplus-fuelband}{Nike Fuelband},
o \href{https://jawbone.com/up}{Jawbone Up}. Estos tipos de dispositivos
son parte del movimiento ``quantified self'' - un grupo de entusiastas
que toman medidas sobre sí mismos regularmente para mejorar su salud,
encontrar patrones en su comportamiento, o porque son geeks
tecnológicos.

Este proyecto utiliza datos de un dispositivo de monitoreo de actividad
personal. Este dispositivo recolecta datos en intervalos de 5 minutos
durante todo el día. Los datos consisten en dos meses de datos de un
individuo anónimo recolectados durante los meses de octubre y noviembre
de 2012 e incluyen el número de pasos tomados en intervalos de 5 minutos
cada día.

\subsection{Carga y Preprocesamiento de
Datos}\label{carga-y-preprocesamiento-de-datos}

\begin{Shaded}
\begin{Highlighting}[]
\FunctionTok{library}\NormalTok{(ggplot2)}
\FunctionTok{library}\NormalTok{(dplyr)}
\FunctionTok{library}\NormalTok{(lubridate)}
\FunctionTok{library}\NormalTok{(lattice)}
\FunctionTok{library}\NormalTok{(knitr)}
\end{Highlighting}
\end{Shaded}

\begin{Shaded}
\begin{Highlighting}[]
\CommentTok{\# Descargar y cargar los datos}
\NormalTok{url }\OtherTok{\textless{}{-}} \StringTok{"https://d396qusza40orc.cloudfront.net/repdata\%2Fdata\%2Factivity.zip"}
\ControlFlowTok{if}\NormalTok{ (}\SpecialCharTok{!}\FunctionTok{file.exists}\NormalTok{(}\StringTok{"activity.csv"}\NormalTok{)) \{}
  \FunctionTok{download.file}\NormalTok{(url, }\AttributeTok{destfile =} \StringTok{"activity.zip"}\NormalTok{)}
  \FunctionTok{unzip}\NormalTok{(}\StringTok{"activity.zip"}\NormalTok{)}
\NormalTok{\}}

\CommentTok{\# Leer los datos}
\NormalTok{activity }\OtherTok{\textless{}{-}} \FunctionTok{read.csv}\NormalTok{(}\StringTok{"activity.csv"}\NormalTok{)}

\CommentTok{\# Convertir la columna date a Date}
\NormalTok{activity}\SpecialCharTok{$}\NormalTok{date }\OtherTok{\textless{}{-}} \FunctionTok{as.Date}\NormalTok{(activity}\SpecialCharTok{$}\NormalTok{date)}

\CommentTok{\# Mostrar la estructura de los datos}
\FunctionTok{str}\NormalTok{(activity)}
\end{Highlighting}
\end{Shaded}

\begin{verbatim}
## 'data.frame':    17568 obs. of  3 variables:
##  $ steps   : int  NA NA NA NA NA NA NA NA NA NA ...
##  $ date    : Date, format: "2012-10-01" "2012-10-01" ...
##  $ interval: int  0 5 10 15 20 25 30 35 40 45 ...
\end{verbatim}

\begin{Shaded}
\begin{Highlighting}[]
\CommentTok{\# Explorar los datos}
\FunctionTok{summary}\NormalTok{(activity)}
\end{Highlighting}
\end{Shaded}

\begin{verbatim}
##      steps             date               interval     
##  Min.   :  0.00   Min.   :2012-10-01   Min.   :   0.0  
##  1st Qu.:  0.00   1st Qu.:2012-10-16   1st Qu.: 588.8  
##  Median :  0.00   Median :2012-10-31   Median :1177.5  
##  Mean   : 37.38   Mean   :2012-10-31   Mean   :1177.5  
##  3rd Qu.: 12.00   3rd Qu.:2012-11-15   3rd Qu.:1766.2  
##  Max.   :806.00   Max.   :2012-11-30   Max.   :2355.0  
##  NA's   :2304
\end{verbatim}

\begin{Shaded}
\begin{Highlighting}[]
\CommentTok{\# Verificar valores faltantes}
\NormalTok{missing\_steps }\OtherTok{\textless{}{-}} \FunctionTok{sum}\NormalTok{(}\FunctionTok{is.na}\NormalTok{(activity}\SpecialCharTok{$}\NormalTok{steps))}
\NormalTok{missing\_percentage }\OtherTok{\textless{}{-}} \FunctionTok{round}\NormalTok{(}\FunctionTok{mean}\NormalTok{(}\FunctionTok{is.na}\NormalTok{(activity}\SpecialCharTok{$}\NormalTok{steps)) }\SpecialCharTok{*} \DecValTok{100}\NormalTok{, }\DecValTok{2}\NormalTok{)}

\FunctionTok{cat}\NormalTok{(}\StringTok{"Total de valores faltantes en \textquotesingle{}steps\textquotesingle{}:"}\NormalTok{, missing\_steps, }\StringTok{"}\SpecialCharTok{\textbackslash{}n}\StringTok{"}\NormalTok{)}
\end{Highlighting}
\end{Shaded}

\begin{verbatim}
## Total de valores faltantes en 'steps': 2304
\end{verbatim}

\begin{Shaded}
\begin{Highlighting}[]
\FunctionTok{cat}\NormalTok{(}\StringTok{"Porcentaje de valores faltantes:"}\NormalTok{, missing\_percentage, }\StringTok{"\%}\SpecialCharTok{\textbackslash{}n}\StringTok{"}\NormalTok{)}
\end{Highlighting}
\end{Shaded}

\begin{verbatim}
## Porcentaje de valores faltantes: 13.11 %
\end{verbatim}

Los datos contienen 17568 observaciones de 3 variables. La variable
\texttt{steps} tiene 2304 valores faltantes (13.11\% del total).

\subsection{¿Cuál es el número total medio de pasos dados por
día?}\label{cuuxe1l-es-el-nuxfamero-total-medio-de-pasos-dados-por-duxeda}

\begin{Shaded}
\begin{Highlighting}[]
\CommentTok{\# Calcular el total de pasos por día (ignorando NAs)}
\NormalTok{daily\_steps }\OtherTok{\textless{}{-}}\NormalTok{ activity }\SpecialCharTok{\%\textgreater{}\%}
  \FunctionTok{group\_by}\NormalTok{(date) }\SpecialCharTok{\%\textgreater{}\%}
  \FunctionTok{summarise}\NormalTok{(}\AttributeTok{total\_steps =} \FunctionTok{sum}\NormalTok{(steps, }\AttributeTok{na.rm =} \ConstantTok{TRUE}\NormalTok{), }\AttributeTok{.groups =} \StringTok{\textquotesingle{}drop\textquotesingle{}}\NormalTok{)}

\CommentTok{\# Remover días con 0 pasos (días con todos los valores NA)}
\NormalTok{daily\_steps\_clean }\OtherTok{\textless{}{-}}\NormalTok{ daily\_steps[daily\_steps}\SpecialCharTok{$}\NormalTok{total\_steps }\SpecialCharTok{\textgreater{}} \DecValTok{0}\NormalTok{, ]}

\CommentTok{\# Estadísticas descriptivas}
\NormalTok{mean\_steps }\OtherTok{\textless{}{-}} \FunctionTok{mean}\NormalTok{(daily\_steps\_clean}\SpecialCharTok{$}\NormalTok{total\_steps)}
\NormalTok{median\_steps }\OtherTok{\textless{}{-}} \FunctionTok{median}\NormalTok{(daily\_steps\_clean}\SpecialCharTok{$}\NormalTok{total\_steps)}

\FunctionTok{cat}\NormalTok{(}\StringTok{"Media de pasos por día:"}\NormalTok{, }\FunctionTok{round}\NormalTok{(mean\_steps, }\DecValTok{2}\NormalTok{), }\StringTok{"}\SpecialCharTok{\textbackslash{}n}\StringTok{"}\NormalTok{)}
\end{Highlighting}
\end{Shaded}

\begin{verbatim}
## Media de pasos por día: 10766.19
\end{verbatim}

\begin{Shaded}
\begin{Highlighting}[]
\FunctionTok{cat}\NormalTok{(}\StringTok{"Mediana de pasos por día:"}\NormalTok{, }\FunctionTok{round}\NormalTok{(median\_steps, }\DecValTok{2}\NormalTok{), }\StringTok{"}\SpecialCharTok{\textbackslash{}n}\StringTok{"}\NormalTok{)}
\end{Highlighting}
\end{Shaded}

\begin{verbatim}
## Mediana de pasos por día: 10765
\end{verbatim}

\begin{Shaded}
\begin{Highlighting}[]
\CommentTok{\# Histograma del total de pasos por día}
\FunctionTok{ggplot}\NormalTok{(daily\_steps\_clean, }\FunctionTok{aes}\NormalTok{(}\AttributeTok{x =}\NormalTok{ total\_steps)) }\SpecialCharTok{+}
  \FunctionTok{geom\_histogram}\NormalTok{(}\AttributeTok{binwidth =} \DecValTok{1000}\NormalTok{, }\AttributeTok{fill =} \StringTok{"steelblue"}\NormalTok{, }\AttributeTok{color =} \StringTok{"black"}\NormalTok{, }\AttributeTok{alpha =} \FloatTok{0.7}\NormalTok{) }\SpecialCharTok{+}
  \FunctionTok{labs}\NormalTok{(}\AttributeTok{title =} \StringTok{"Histograma del Número Total de Pasos por Día (Datos Originales)"}\NormalTok{,}
       \AttributeTok{x =} \StringTok{"Total de Pasos por Día"}\NormalTok{,}
       \AttributeTok{y =} \StringTok{"Frecuencia"}\NormalTok{) }\SpecialCharTok{+}
  \FunctionTok{theme\_minimal}\NormalTok{() }\SpecialCharTok{+}
  \FunctionTok{geom\_vline}\NormalTok{(}\FunctionTok{aes}\NormalTok{(}\AttributeTok{xintercept =}\NormalTok{ mean\_steps), }\AttributeTok{color =} \StringTok{"red"}\NormalTok{, }\AttributeTok{linetype =} \StringTok{"dashed"}\NormalTok{, }\AttributeTok{size =} \DecValTok{1}\NormalTok{) }\SpecialCharTok{+}
  \FunctionTok{geom\_vline}\NormalTok{(}\FunctionTok{aes}\NormalTok{(}\AttributeTok{xintercept =}\NormalTok{ median\_steps), }\AttributeTok{color =} \StringTok{"blue"}\NormalTok{, }\AttributeTok{linetype =} \StringTok{"dashed"}\NormalTok{, }\AttributeTok{size =} \DecValTok{1}\NormalTok{) }\SpecialCharTok{+}
  \FunctionTok{annotate}\NormalTok{(}\StringTok{"text"}\NormalTok{, }\AttributeTok{x =}\NormalTok{ mean\_steps }\SpecialCharTok{+} \DecValTok{1500}\NormalTok{, }\AttributeTok{y =} \DecValTok{8}\NormalTok{, }
           \AttributeTok{label =} \FunctionTok{paste}\NormalTok{(}\StringTok{"Media ="}\NormalTok{, }\FunctionTok{round}\NormalTok{(mean\_steps, }\DecValTok{0}\NormalTok{)), }\AttributeTok{color =} \StringTok{"red"}\NormalTok{) }\SpecialCharTok{+}
  \FunctionTok{annotate}\NormalTok{(}\StringTok{"text"}\NormalTok{, }\AttributeTok{x =}\NormalTok{ median\_steps }\SpecialCharTok{+} \DecValTok{1500}\NormalTok{, }\AttributeTok{y =} \DecValTok{7}\NormalTok{, }
           \AttributeTok{label =} \FunctionTok{paste}\NormalTok{(}\StringTok{"Mediana ="}\NormalTok{, }\FunctionTok{round}\NormalTok{(median\_steps, }\DecValTok{0}\NormalTok{)), }\AttributeTok{color =} \StringTok{"blue"}\NormalTok{)}
\end{Highlighting}
\end{Shaded}

\pandocbounded{\includegraphics[keepaspectratio]{Markdown-Script_files/figure-latex/histogram_original-1.pdf}}

La media del número total de pasos por día es
\textbf{\ensuremath{1.076619\times 10^{4}}} y la mediana es
\textbf{\ensuremath{1.0765\times 10^{4}}}.

\subsection{¿Cuál es el patrón de actividad diaria
promedio?}\label{cuuxe1l-es-el-patruxf3n-de-actividad-diaria-promedio}

\begin{Shaded}
\begin{Highlighting}[]
\CommentTok{\# Calcular el promedio de pasos por intervalo de 5 minutos}
\NormalTok{interval\_pattern }\OtherTok{\textless{}{-}}\NormalTok{ activity }\SpecialCharTok{\%\textgreater{}\%}
  \FunctionTok{filter}\NormalTok{(}\SpecialCharTok{!}\FunctionTok{is.na}\NormalTok{(steps)) }\SpecialCharTok{\%\textgreater{}\%}
  \FunctionTok{group\_by}\NormalTok{(interval) }\SpecialCharTok{\%\textgreater{}\%}
  \FunctionTok{summarise}\NormalTok{(}\AttributeTok{avg\_steps =} \FunctionTok{mean}\NormalTok{(steps), }\AttributeTok{.groups =} \StringTok{\textquotesingle{}drop\textquotesingle{}}\NormalTok{)}

\CommentTok{\# Encontrar el intervalo con el máximo número de pasos}
\NormalTok{max\_interval }\OtherTok{\textless{}{-}}\NormalTok{ interval\_pattern}\SpecialCharTok{$}\NormalTok{interval[}\FunctionTok{which.max}\NormalTok{(interval\_pattern}\SpecialCharTok{$}\NormalTok{avg\_steps)]}
\NormalTok{max\_steps\_avg }\OtherTok{\textless{}{-}} \FunctionTok{max}\NormalTok{(interval\_pattern}\SpecialCharTok{$}\NormalTok{avg\_steps)}

\CommentTok{\# Función para convertir intervalo a formato de tiempo}
\NormalTok{interval\_to\_time }\OtherTok{\textless{}{-}} \ControlFlowTok{function}\NormalTok{(interval) \{}
\NormalTok{  hours }\OtherTok{\textless{}{-}}\NormalTok{ interval }\SpecialCharTok{\%/\%} \DecValTok{100}
\NormalTok{  minutes }\OtherTok{\textless{}{-}}\NormalTok{ interval }\SpecialCharTok{\%\%} \DecValTok{100}
  \FunctionTok{sprintf}\NormalTok{(}\StringTok{"\%02d:\%02d"}\NormalTok{, hours, minutes)}
\NormalTok{\}}

\FunctionTok{cat}\NormalTok{(}\StringTok{"Intervalo con máximo número de pasos:"}\NormalTok{, max\_interval, }\StringTok{"}\SpecialCharTok{\textbackslash{}n}\StringTok{"}\NormalTok{)}
\end{Highlighting}
\end{Shaded}

\begin{verbatim}
## Intervalo con máximo número de pasos: 835
\end{verbatim}

\begin{Shaded}
\begin{Highlighting}[]
\FunctionTok{cat}\NormalTok{(}\StringTok{"Promedio de pasos en ese intervalo:"}\NormalTok{, }\FunctionTok{round}\NormalTok{(max\_steps\_avg, }\DecValTok{2}\NormalTok{), }\StringTok{"}\SpecialCharTok{\textbackslash{}n}\StringTok{"}\NormalTok{)}
\end{Highlighting}
\end{Shaded}

\begin{verbatim}
## Promedio de pasos en ese intervalo: 206.17
\end{verbatim}

\begin{Shaded}
\begin{Highlighting}[]
\FunctionTok{cat}\NormalTok{(}\StringTok{"Hora del día con máxima actividad:"}\NormalTok{, }\FunctionTok{interval\_to\_time}\NormalTok{(max\_interval), }\StringTok{"}\SpecialCharTok{\textbackslash{}n}\StringTok{"}\NormalTok{)}
\end{Highlighting}
\end{Shaded}

\begin{verbatim}
## Hora del día con máxima actividad: 08:35
\end{verbatim}

\begin{Shaded}
\begin{Highlighting}[]
\CommentTok{\# Gráfico de series de tiempo del patrón de actividad diaria}
\FunctionTok{ggplot}\NormalTok{(interval\_pattern, }\FunctionTok{aes}\NormalTok{(}\AttributeTok{x =}\NormalTok{ interval, }\AttributeTok{y =}\NormalTok{ avg\_steps)) }\SpecialCharTok{+}
  \FunctionTok{geom\_line}\NormalTok{(}\AttributeTok{color =} \StringTok{"blue"}\NormalTok{, }\AttributeTok{size =} \DecValTok{1}\NormalTok{) }\SpecialCharTok{+}
  \FunctionTok{labs}\NormalTok{(}\AttributeTok{title =} \StringTok{"Patrón de Actividad Diaria Promedio"}\NormalTok{,}
       \AttributeTok{x =} \StringTok{"Intervalo de 5 minutos"}\NormalTok{,}
       \AttributeTok{y =} \StringTok{"Número Promedio de Pasos"}\NormalTok{) }\SpecialCharTok{+}
  \FunctionTok{theme\_minimal}\NormalTok{() }\SpecialCharTok{+}
  \FunctionTok{geom\_vline}\NormalTok{(}\FunctionTok{aes}\NormalTok{(}\AttributeTok{xintercept =}\NormalTok{ max\_interval), }\AttributeTok{color =} \StringTok{"red"}\NormalTok{, }\AttributeTok{linetype =} \StringTok{"dashed"}\NormalTok{) }\SpecialCharTok{+}
  \FunctionTok{annotate}\NormalTok{(}\StringTok{"text"}\NormalTok{, }\AttributeTok{x =}\NormalTok{ max\_interval }\SpecialCharTok{+} \DecValTok{200}\NormalTok{, }\AttributeTok{y =}\NormalTok{ max\_steps\_avg }\SpecialCharTok{{-}} \DecValTok{20}\NormalTok{, }
           \AttributeTok{label =} \FunctionTok{paste}\NormalTok{(}\StringTok{"Máximo:"}\NormalTok{, }\FunctionTok{interval\_to\_time}\NormalTok{(max\_interval)), }\AttributeTok{color =} \StringTok{"red"}\NormalTok{)}
\end{Highlighting}
\end{Shaded}

\pandocbounded{\includegraphics[keepaspectratio]{Markdown-Script_files/figure-latex/time_series_plot-1.pdf}}

El intervalo de 5 minutos que, en promedio, contiene el máximo número de
pasos es \textbf{835} (las 08:35), con un promedio de \textbf{206.17}
pasos.

\subsection{Imputación de Valores
Faltantes}\label{imputaciuxf3n-de-valores-faltantes}

\subsubsection{Análisis de Valores
Faltantes}\label{anuxe1lisis-de-valores-faltantes}

\begin{Shaded}
\begin{Highlighting}[]
\CommentTok{\# Analizar el patrón de valores faltantes}
\NormalTok{missing\_pattern }\OtherTok{\textless{}{-}}\NormalTok{ activity }\SpecialCharTok{\%\textgreater{}\%}
  \FunctionTok{group\_by}\NormalTok{(date) }\SpecialCharTok{\%\textgreater{}\%}
  \FunctionTok{summarise}\NormalTok{(}\AttributeTok{missing\_count =} \FunctionTok{sum}\NormalTok{(}\FunctionTok{is.na}\NormalTok{(steps)),}
            \AttributeTok{total\_intervals =} \FunctionTok{n}\NormalTok{(),}
            \AttributeTok{.groups =} \StringTok{\textquotesingle{}drop\textquotesingle{}}\NormalTok{)}

\CommentTok{\# Días con valores faltantes}
\NormalTok{days\_with\_missing }\OtherTok{\textless{}{-}}\NormalTok{ missing\_pattern[missing\_pattern}\SpecialCharTok{$}\NormalTok{missing\_count }\SpecialCharTok{\textgreater{}} \DecValTok{0}\NormalTok{, ]}
\FunctionTok{kable}\NormalTok{(days\_with\_missing, }\AttributeTok{caption =} \StringTok{"Días con Valores Faltantes"}\NormalTok{)}
\end{Highlighting}
\end{Shaded}

\begin{longtable}[]{@{}lrr@{}}
\caption{Días con Valores Faltantes}\tabularnewline
\toprule\noalign{}
date & missing\_count & total\_intervals \\
\midrule\noalign{}
\endfirsthead
\toprule\noalign{}
date & missing\_count & total\_intervals \\
\midrule\noalign{}
\endhead
\bottomrule\noalign{}
\endlastfoot
2012-10-01 & 288 & 288 \\
2012-10-08 & 288 & 288 \\
2012-11-01 & 288 & 288 \\
2012-11-04 & 288 & 288 \\
2012-11-09 & 288 & 288 \\
2012-11-10 & 288 & 288 \\
2012-11-14 & 288 & 288 \\
2012-11-30 & 288 & 288 \\
\end{longtable}

\begin{Shaded}
\begin{Highlighting}[]
\NormalTok{complete\_missing\_days }\OtherTok{\textless{}{-}} \FunctionTok{sum}\NormalTok{(days\_with\_missing}\SpecialCharTok{$}\NormalTok{missing\_count }\SpecialCharTok{==} \DecValTok{288}\NormalTok{)}
\FunctionTok{cat}\NormalTok{(}\StringTok{"Días con TODOS los valores faltantes:"}\NormalTok{, complete\_missing\_days, }\StringTok{"}\SpecialCharTok{\textbackslash{}n}\StringTok{"}\NormalTok{)}
\end{Highlighting}
\end{Shaded}

\begin{verbatim}
## Días con TODOS los valores faltantes: 8
\end{verbatim}

\subsubsection{Estrategia de
Imputación}\label{estrategia-de-imputaciuxf3n}

\textbf{Estrategia elegida}: Reemplazar cada valor faltante con el
promedio de pasos para ese intervalo de 5 minutos específico.

\textbf{Justificación}: 1. Los días con valores faltantes tienen TODOS
los intervalos faltantes (días completos) 2. No podemos usar el promedio
del día porque no hay datos para esos días 3. Usar el promedio del
intervalo captura mejor los patrones de actividad diaria 4. Es más
realista que usar un valor global o cero

\begin{Shaded}
\begin{Highlighting}[]
\CommentTok{\# Implementar la estrategia de imputación}
\NormalTok{activity\_imputed }\OtherTok{\textless{}{-}}\NormalTok{ activity}

\CommentTok{\# Crear un vector de medias por intervalo para imputación eficiente}
\NormalTok{interval\_means }\OtherTok{\textless{}{-}}\NormalTok{ interval\_pattern}\SpecialCharTok{$}\NormalTok{avg\_steps}
\FunctionTok{names}\NormalTok{(interval\_means) }\OtherTok{\textless{}{-}}\NormalTok{ interval\_pattern}\SpecialCharTok{$}\NormalTok{interval}

\CommentTok{\# Imputar valores faltantes}
\ControlFlowTok{for}\NormalTok{ (i }\ControlFlowTok{in} \DecValTok{1}\SpecialCharTok{:}\FunctionTok{nrow}\NormalTok{(activity\_imputed)) \{}
  \ControlFlowTok{if}\NormalTok{ (}\FunctionTok{is.na}\NormalTok{(activity\_imputed}\SpecialCharTok{$}\NormalTok{steps[i])) \{}
\NormalTok{    interval\_value }\OtherTok{\textless{}{-}} \FunctionTok{as.character}\NormalTok{(activity\_imputed}\SpecialCharTok{$}\NormalTok{interval[i])}
\NormalTok{    activity\_imputed}\SpecialCharTok{$}\NormalTok{steps[i] }\OtherTok{\textless{}{-}}\NormalTok{ interval\_means[interval\_value]}
\NormalTok{  \}}
\NormalTok{\}}

\CommentTok{\# Verificar que no hay valores faltantes}
\FunctionTok{cat}\NormalTok{(}\StringTok{"Valores faltantes después de la imputación:"}\NormalTok{, }\FunctionTok{sum}\NormalTok{(}\FunctionTok{is.na}\NormalTok{(activity\_imputed}\SpecialCharTok{$}\NormalTok{steps)), }\StringTok{"}\SpecialCharTok{\textbackslash{}n}\StringTok{"}\NormalTok{)}
\end{Highlighting}
\end{Shaded}

\begin{verbatim}
## Valores faltantes después de la imputación: 0
\end{verbatim}

\subsubsection{Impacto de la
Imputación}\label{impacto-de-la-imputaciuxf3n}

\begin{Shaded}
\begin{Highlighting}[]
\CommentTok{\# Calcular nuevas estadísticas con datos imputados}
\NormalTok{daily\_steps\_imputed }\OtherTok{\textless{}{-}}\NormalTok{ activity\_imputed }\SpecialCharTok{\%\textgreater{}\%}
  \FunctionTok{group\_by}\NormalTok{(date) }\SpecialCharTok{\%\textgreater{}\%}
  \FunctionTok{summarise}\NormalTok{(}\AttributeTok{total\_steps =} \FunctionTok{sum}\NormalTok{(steps), }\AttributeTok{.groups =} \StringTok{\textquotesingle{}drop\textquotesingle{}}\NormalTok{)}

\NormalTok{mean\_imputed }\OtherTok{\textless{}{-}} \FunctionTok{mean}\NormalTok{(daily\_steps\_imputed}\SpecialCharTok{$}\NormalTok{total\_steps)}
\NormalTok{median\_imputed }\OtherTok{\textless{}{-}} \FunctionTok{median}\NormalTok{(daily\_steps\_imputed}\SpecialCharTok{$}\NormalTok{total\_steps)}

\CommentTok{\# Comparación de estadísticas}
\NormalTok{comparison }\OtherTok{\textless{}{-}} \FunctionTok{data.frame}\NormalTok{(}
\NormalTok{  Estadística }\OtherTok{=} \FunctionTok{c}\NormalTok{(}\StringTok{"Media"}\NormalTok{, }\StringTok{"Mediana"}\NormalTok{),}
  \AttributeTok{Original =} \FunctionTok{c}\NormalTok{(}\FunctionTok{round}\NormalTok{(mean\_steps, }\DecValTok{2}\NormalTok{), }\FunctionTok{round}\NormalTok{(median\_steps, }\DecValTok{2}\NormalTok{)),}
  \AttributeTok{Imputado =} \FunctionTok{c}\NormalTok{(}\FunctionTok{round}\NormalTok{(mean\_imputed, }\DecValTok{2}\NormalTok{), }\FunctionTok{round}\NormalTok{(median\_imputed, }\DecValTok{2}\NormalTok{)),}
  \AttributeTok{Diferencia =} \FunctionTok{c}\NormalTok{(}\FunctionTok{round}\NormalTok{(mean\_imputed }\SpecialCharTok{{-}}\NormalTok{ mean\_steps, }\DecValTok{2}\NormalTok{), }
                \FunctionTok{round}\NormalTok{(median\_imputed }\SpecialCharTok{{-}}\NormalTok{ median\_steps, }\DecValTok{2}\NormalTok{))}
\NormalTok{)}

\FunctionTok{kable}\NormalTok{(comparison, }\AttributeTok{caption =} \StringTok{"Comparación de Estadísticas: Antes vs Después de Imputación"}\NormalTok{)}
\end{Highlighting}
\end{Shaded}

\begin{longtable}[]{@{}lrrr@{}}
\caption{Comparación de Estadísticas: Antes vs Después de
Imputación}\tabularnewline
\toprule\noalign{}
Estadística & Original & Imputado & Diferencia \\
\midrule\noalign{}
\endfirsthead
\toprule\noalign{}
Estadística & Original & Imputado & Diferencia \\
\midrule\noalign{}
\endhead
\bottomrule\noalign{}
\endlastfoot
Media & 10766.19 & 10766.19 & 0.00 \\
Mediana & 10765.00 & 10766.19 & 1.19 \\
\end{longtable}

\begin{Shaded}
\begin{Highlighting}[]
\CommentTok{\# Histograma con datos imputados}
\FunctionTok{ggplot}\NormalTok{(daily\_steps\_imputed, }\FunctionTok{aes}\NormalTok{(}\AttributeTok{x =}\NormalTok{ total\_steps)) }\SpecialCharTok{+}
  \FunctionTok{geom\_histogram}\NormalTok{(}\AttributeTok{binwidth =} \DecValTok{1000}\NormalTok{, }\AttributeTok{fill =} \StringTok{"lightcoral"}\NormalTok{, }\AttributeTok{color =} \StringTok{"black"}\NormalTok{, }\AttributeTok{alpha =} \FloatTok{0.7}\NormalTok{) }\SpecialCharTok{+}
  \FunctionTok{labs}\NormalTok{(}\AttributeTok{title =} \StringTok{"Histograma del Número Total de Pasos por Día (Datos Imputados)"}\NormalTok{,}
       \AttributeTok{x =} \StringTok{"Total de Pasos por Día"}\NormalTok{,}
       \AttributeTok{y =} \StringTok{"Frecuencia"}\NormalTok{) }\SpecialCharTok{+}
  \FunctionTok{theme\_minimal}\NormalTok{() }\SpecialCharTok{+}
  \FunctionTok{geom\_vline}\NormalTok{(}\FunctionTok{aes}\NormalTok{(}\AttributeTok{xintercept =}\NormalTok{ mean\_imputed), }\AttributeTok{color =} \StringTok{"red"}\NormalTok{, }\AttributeTok{linetype =} \StringTok{"dashed"}\NormalTok{, }\AttributeTok{size =} \DecValTok{1}\NormalTok{) }\SpecialCharTok{+}
  \FunctionTok{geom\_vline}\NormalTok{(}\FunctionTok{aes}\NormalTok{(}\AttributeTok{xintercept =}\NormalTok{ median\_imputed), }\AttributeTok{color =} \StringTok{"blue"}\NormalTok{, }\AttributeTok{linetype =} \StringTok{"dashed"}\NormalTok{, }\AttributeTok{size =} \DecValTok{1}\NormalTok{) }\SpecialCharTok{+}
  \FunctionTok{annotate}\NormalTok{(}\StringTok{"text"}\NormalTok{, }\AttributeTok{x =}\NormalTok{ mean\_imputed }\SpecialCharTok{+} \DecValTok{1500}\NormalTok{, }\AttributeTok{y =} \DecValTok{12}\NormalTok{, }
           \AttributeTok{label =} \FunctionTok{paste}\NormalTok{(}\StringTok{"Media ="}\NormalTok{, }\FunctionTok{round}\NormalTok{(mean\_imputed, }\DecValTok{0}\NormalTok{)), }\AttributeTok{color =} \StringTok{"red"}\NormalTok{) }\SpecialCharTok{+}
  \FunctionTok{annotate}\NormalTok{(}\StringTok{"text"}\NormalTok{, }\AttributeTok{x =}\NormalTok{ median\_imputed }\SpecialCharTok{+} \DecValTok{1500}\NormalTok{, }\AttributeTok{y =} \DecValTok{11}\NormalTok{, }
           \AttributeTok{label =} \FunctionTok{paste}\NormalTok{(}\StringTok{"Mediana ="}\NormalTok{, }\FunctionTok{round}\NormalTok{(median\_imputed, }\DecValTok{0}\NormalTok{)), }\AttributeTok{color =} \StringTok{"blue"}\NormalTok{)}
\end{Highlighting}
\end{Shaded}

\pandocbounded{\includegraphics[keepaspectratio]{Markdown-Script_files/figure-latex/histogram_imputed-1.pdf}}

\textbf{Impacto de la imputación}: - La media aumentó de
\ensuremath{1.076619\times 10^{4}} a \ensuremath{1.076619\times 10^{4}}
pasos (+0) - La mediana cambió de \ensuremath{1.0765\times 10^{4}} a
\ensuremath{1.076619\times 10^{4}} pasos (+1.19)

\subsection{¿Hay diferencias en los patrones de actividad entre días de
semana y fines de
semana?}\label{hay-diferencias-en-los-patrones-de-actividad-entre-duxedas-de-semana-y-fines-de-semana}

\begin{Shaded}
\begin{Highlighting}[]
\CommentTok{\# Crear factor para tipo de día (usando datos imputados)}
\NormalTok{activity\_imputed}\SpecialCharTok{$}\NormalTok{day\_type }\OtherTok{\textless{}{-}} \FunctionTok{ifelse}\NormalTok{(}\FunctionTok{weekdays}\NormalTok{(activity\_imputed}\SpecialCharTok{$}\NormalTok{date) }\SpecialCharTok{\%in\%} \FunctionTok{c}\NormalTok{(}\StringTok{"Saturday"}\NormalTok{, }\StringTok{"Sunday"}\NormalTok{),}
                                   \StringTok{"weekend"}\NormalTok{, }\StringTok{"weekday"}\NormalTok{)}
\NormalTok{activity\_imputed}\SpecialCharTok{$}\NormalTok{day\_type }\OtherTok{\textless{}{-}} \FunctionTok{as.factor}\NormalTok{(activity\_imputed}\SpecialCharTok{$}\NormalTok{day\_type)}

\CommentTok{\# Calcular patrones de actividad por tipo de día}
\NormalTok{activity\_patterns }\OtherTok{\textless{}{-}}\NormalTok{ activity\_imputed }\SpecialCharTok{\%\textgreater{}\%}
  \FunctionTok{group\_by}\NormalTok{(interval, day\_type) }\SpecialCharTok{\%\textgreater{}\%}
  \FunctionTok{summarise}\NormalTok{(}\AttributeTok{avg\_steps =} \FunctionTok{mean}\NormalTok{(steps), }\AttributeTok{.groups =} \StringTok{\textquotesingle{}drop\textquotesingle{}}\NormalTok{)}

\CommentTok{\# Estadísticas por tipo de día}
\NormalTok{daily\_by\_type }\OtherTok{\textless{}{-}}\NormalTok{ activity\_imputed }\SpecialCharTok{\%\textgreater{}\%}
  \FunctionTok{group\_by}\NormalTok{(date, day\_type) }\SpecialCharTok{\%\textgreater{}\%}
  \FunctionTok{summarise}\NormalTok{(}\AttributeTok{total\_steps =} \FunctionTok{sum}\NormalTok{(steps), }\AttributeTok{.groups =} \StringTok{\textquotesingle{}drop\textquotesingle{}}\NormalTok{) }\SpecialCharTok{\%\textgreater{}\%}
  \FunctionTok{group\_by}\NormalTok{(day\_type) }\SpecialCharTok{\%\textgreater{}\%}
  \FunctionTok{summarise}\NormalTok{(}\AttributeTok{mean\_steps =} \FunctionTok{round}\NormalTok{(}\FunctionTok{mean}\NormalTok{(total\_steps), }\DecValTok{2}\NormalTok{),}
            \AttributeTok{median\_steps =} \FunctionTok{round}\NormalTok{(}\FunctionTok{median}\NormalTok{(total\_steps), }\DecValTok{2}\NormalTok{),}
            \AttributeTok{.groups =} \StringTok{\textquotesingle{}drop\textquotesingle{}}\NormalTok{)}

\FunctionTok{kable}\NormalTok{(daily\_by\_type, }\AttributeTok{caption =} \StringTok{"Estadísticas de Pasos por Tipo de Día"}\NormalTok{)}
\end{Highlighting}
\end{Shaded}

\begin{longtable}[]{@{}lrr@{}}
\caption{Estadísticas de Pasos por Tipo de Día}\tabularnewline
\toprule\noalign{}
day\_type & mean\_steps & median\_steps \\
\midrule\noalign{}
\endfirsthead
\toprule\noalign{}
day\_type & mean\_steps & median\_steps \\
\midrule\noalign{}
\endhead
\bottomrule\noalign{}
\endlastfoot
weekday & 10766.19 & 10766.19 \\
\end{longtable}

\begin{Shaded}
\begin{Highlighting}[]
\CommentTok{\# Panel plot comparando weekdays vs weekends usando ggplot2}
\FunctionTok{ggplot}\NormalTok{(activity\_patterns, }\FunctionTok{aes}\NormalTok{(}\AttributeTok{x =}\NormalTok{ interval, }\AttributeTok{y =}\NormalTok{ avg\_steps)) }\SpecialCharTok{+}
  \FunctionTok{geom\_line}\NormalTok{(}\AttributeTok{color =} \StringTok{"blue"}\NormalTok{, }\AttributeTok{size =} \DecValTok{1}\NormalTok{) }\SpecialCharTok{+}
  \FunctionTok{facet\_wrap}\NormalTok{(}\SpecialCharTok{\textasciitilde{}}\NormalTok{ day\_type, }\AttributeTok{ncol =} \DecValTok{1}\NormalTok{, }
             \AttributeTok{labeller =} \FunctionTok{labeller}\NormalTok{(}\AttributeTok{day\_type =} \FunctionTok{c}\NormalTok{(}\StringTok{"weekday"} \OtherTok{=} \StringTok{"Días de Semana"}\NormalTok{, }
                                             \StringTok{"weekend"} \OtherTok{=} \StringTok{"Fines de Semana"}\NormalTok{))) }\SpecialCharTok{+}
  \FunctionTok{labs}\NormalTok{(}\AttributeTok{title =} \StringTok{"Comparación de Patrones de Actividad: Días de Semana vs Fines de Semana"}\NormalTok{,}
       \AttributeTok{x =} \StringTok{"Intervalo de 5 minutos"}\NormalTok{,}
       \AttributeTok{y =} \StringTok{"Número Promedio de Pasos"}\NormalTok{) }\SpecialCharTok{+}
  \FunctionTok{theme\_minimal}\NormalTok{() }\SpecialCharTok{+}
  \FunctionTok{theme}\NormalTok{(}\AttributeTok{strip.text =} \FunctionTok{element\_text}\NormalTok{(}\AttributeTok{size =} \DecValTok{12}\NormalTok{, }\AttributeTok{face =} \StringTok{"bold"}\NormalTok{))}
\end{Highlighting}
\end{Shaded}

\pandocbounded{\includegraphics[keepaspectratio]{Markdown-Script_files/figure-latex/panel_plot-1.pdf}}

\begin{Shaded}
\begin{Highlighting}[]
\CommentTok{\# Panel plot usando lattice (como en el ejemplo original)}
\FunctionTok{library}\NormalTok{(lattice)}
\FunctionTok{xyplot}\NormalTok{(avg\_steps }\SpecialCharTok{\textasciitilde{}}\NormalTok{ interval }\SpecialCharTok{|}\NormalTok{ day\_type, }\AttributeTok{data =}\NormalTok{ activity\_patterns, }
       \AttributeTok{type =} \StringTok{"l"}\NormalTok{, }\AttributeTok{layout =} \FunctionTok{c}\NormalTok{(}\DecValTok{1}\NormalTok{, }\DecValTok{2}\NormalTok{),}
       \AttributeTok{xlab =} \StringTok{"Intervalo de 5 minutos"}\NormalTok{, }
       \AttributeTok{ylab =} \StringTok{"Número Promedio de Pasos"}\NormalTok{,}
       \AttributeTok{main =} \StringTok{"Patrones de Actividad: Días de Semana vs Fines de Semana"}\NormalTok{,}
       \AttributeTok{strip =} \FunctionTok{strip.custom}\NormalTok{(}\AttributeTok{factor.levels =} \FunctionTok{c}\NormalTok{(}\StringTok{"Días de Semana"}\NormalTok{, }\StringTok{"Fines de Semana"}\NormalTok{)))}
\end{Highlighting}
\end{Shaded}

\pandocbounded{\includegraphics[keepaspectratio]{Markdown-Script_files/figure-latex/panel_plot_lattice-1.pdf}}

\subsubsection{Análisis de
Diferencias}\label{anuxe1lisis-de-diferencias}

\begin{Shaded}
\begin{Highlighting}[]
\CommentTok{\# Análisis más detallado de las diferencias}
\NormalTok{weekday\_pattern }\OtherTok{\textless{}{-}}\NormalTok{ activity\_patterns[activity\_patterns}\SpecialCharTok{$}\NormalTok{day\_type }\SpecialCharTok{==} \StringTok{"weekday"}\NormalTok{, ]}
\NormalTok{weekend\_pattern }\OtherTok{\textless{}{-}}\NormalTok{ activity\_patterns[activity\_patterns}\SpecialCharTok{$}\NormalTok{day\_type }\SpecialCharTok{==} \StringTok{"weekend"}\NormalTok{, ]}

\CommentTok{\# Picos de actividad}
\NormalTok{weekday\_max }\OtherTok{\textless{}{-}}\NormalTok{ weekday\_pattern}\SpecialCharTok{$}\NormalTok{interval[}\FunctionTok{which.max}\NormalTok{(weekday\_pattern}\SpecialCharTok{$}\NormalTok{avg\_steps)]}
\NormalTok{weekend\_max }\OtherTok{\textless{}{-}}\NormalTok{ weekend\_pattern}\SpecialCharTok{$}\NormalTok{interval[}\FunctionTok{which.max}\NormalTok{(weekend\_pattern}\SpecialCharTok{$}\NormalTok{avg\_steps)]}

\FunctionTok{cat}\NormalTok{(}\StringTok{"Pico de actividad en días de semana:"}\NormalTok{, }\FunctionTok{interval\_to\_time}\NormalTok{(weekday\_max), }
    \StringTok{"con"}\NormalTok{, }\FunctionTok{round}\NormalTok{(}\FunctionTok{max}\NormalTok{(weekday\_pattern}\SpecialCharTok{$}\NormalTok{avg\_steps), }\DecValTok{2}\NormalTok{), }\StringTok{"pasos}\SpecialCharTok{\textbackslash{}n}\StringTok{"}\NormalTok{)}
\end{Highlighting}
\end{Shaded}

\begin{verbatim}
## Pico de actividad en días de semana: 08:35 con 206.17 pasos
\end{verbatim}

\begin{Shaded}
\begin{Highlighting}[]
\FunctionTok{cat}\NormalTok{(}\StringTok{"Pico de actividad en fines de semana:"}\NormalTok{, }\FunctionTok{interval\_to\_time}\NormalTok{(weekend\_max), }
    \StringTok{"con"}\NormalTok{, }\FunctionTok{round}\NormalTok{(}\FunctionTok{max}\NormalTok{(weekend\_pattern}\SpecialCharTok{$}\NormalTok{avg\_steps), }\DecValTok{2}\NormalTok{), }\StringTok{"pasos}\SpecialCharTok{\textbackslash{}n}\StringTok{"}\NormalTok{)}
\end{Highlighting}
\end{Shaded}

\begin{verbatim}
## Pico de actividad en fines de semana:  con -Inf pasos
\end{verbatim}

\begin{Shaded}
\begin{Highlighting}[]
\CommentTok{\# Promedios generales}
\NormalTok{avg\_weekday }\OtherTok{\textless{}{-}} \FunctionTok{mean}\NormalTok{(weekday\_pattern}\SpecialCharTok{$}\NormalTok{avg\_steps)}
\NormalTok{avg\_weekend }\OtherTok{\textless{}{-}} \FunctionTok{mean}\NormalTok{(weekend\_pattern}\SpecialCharTok{$}\NormalTok{avg\_steps)}

\FunctionTok{cat}\NormalTok{(}\StringTok{"Promedio general de pasos por intervalo:}\SpecialCharTok{\textbackslash{}n}\StringTok{"}\NormalTok{)}
\end{Highlighting}
\end{Shaded}

\begin{verbatim}
## Promedio general de pasos por intervalo:
\end{verbatim}

\begin{Shaded}
\begin{Highlighting}[]
\FunctionTok{cat}\NormalTok{(}\StringTok{"Días de semana:"}\NormalTok{, }\FunctionTok{round}\NormalTok{(avg\_weekday, }\DecValTok{2}\NormalTok{), }\StringTok{"pasos}\SpecialCharTok{\textbackslash{}n}\StringTok{"}\NormalTok{)}
\end{Highlighting}
\end{Shaded}

\begin{verbatim}
## Días de semana: 37.38 pasos
\end{verbatim}

\begin{Shaded}
\begin{Highlighting}[]
\FunctionTok{cat}\NormalTok{(}\StringTok{"Fines de semana:"}\NormalTok{, }\FunctionTok{round}\NormalTok{(avg\_weekend, }\DecValTok{2}\NormalTok{), }\StringTok{"pasos}\SpecialCharTok{\textbackslash{}n}\StringTok{"}\NormalTok{)}
\end{Highlighting}
\end{Shaded}

\begin{verbatim}
## Fines de semana: NaN pasos
\end{verbatim}

\subsection{Conclusiones}\label{conclusiones}

\begin{enumerate}
\def\labelenumi{\arabic{enumi}.}
\item
  \textbf{Actividad diaria total}: El individuo camina en promedio
  \ensuremath{1.0766\times 10^{4}} pasos por día (después de
  imputación).
\item
  \textbf{Patrón de actividad}: La máxima actividad ocurre a las 08:35
  con un promedio de 206.2 pasos.
\item
  \textbf{Impacto de la imputación}: La estrategia de imputación por
  promedio de intervalo aumentó ligeramente las estimaciones (media: +0
  pasos).
\item
  \textbf{Diferencias por tipo de día}:

  \begin{itemize}
  \tightlist
  \item
    Los \textbf{días de semana} muestran un pico de actividad más
    pronunciado en la mañana (08:35)
  \item
    Los \textbf{fines de semana} tienen un patrón más distribuido con
    pico más tardío ()
  \item
    El promedio de actividad es mayor en fines de semana (NaN vs 37.4
    pasos por intervalo)
  \end{itemize}
\end{enumerate}

Este análisis demuestra la importancia de considerar patrones temporales
en datos de actividad física y el impacto de las estrategias de
imputación en las conclusiones del estudio.

\end{document}
